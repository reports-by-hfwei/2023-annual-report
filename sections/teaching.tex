% teaching.tex

%%%%%%%%%%%%%%%%%%%%
\begin{frame}{}
  \fig{width = 0.80\textwidth}{figs/teach}
\end{frame}
%%%%%%%%%%%%%%%%%%%%

%%%%%%%%%%%%%%%%%%%%
\begin{frame}{}
  \begin{table}[t]
    \centering
    \renewcommand\arraystretch{1.2}
    \begin{tabular}[]{c|c|c|c}
      \hline
      学期       & 课程                & 学分    & 课时 \\ \hline \hline
      2023年春季 & 编译原理 (1 班)           & 3  & 54    \\ \hline
      2023年暑期 & 大语言模型原理与应用        & 1  & 2    \\ \hline
      2023年秋季 & C 语言程序设计基础 (1 班)   & 2  & 36    \\ \hline
      2023年秋季 & C 语言程序设计基础 (2 班)   & 2  & 36    \\ \hline
      & & & \blue{\bf 128} \\ \hline
    \end{tabular}
  \end{table}

  \begin{columns}
    \column{0.33\textwidth}
      \fig{width = 0.40\textwidth}{figs/cpl-logo}
    \column{0.35\textwidth}
      \fig{width = 0.50\textwidth}{figs/compiler-logo}
  \end{columns}
\end{frame}
%%%%%%%%%%%%%%%%%%%%

%%%%%%%%%%%%%%%%%%%%
\begin{frame}{}
  \[
    (\underbrace{200}_{\text{{软件学院}}}
      + \underbrace{88}_{\text{\red{\bf 重修、跨专业}}})
      + (\underbrace{98 + 95 + 87 + 90 + 86 + 90}_{\text{技术科学试验班}})
      + \underbrace{32}_{\text{\blue{\bf 技科重修班}}} = 866 \text{ 名学生}
  \]

  \fig{width = 0.25\textwidth}{figs/cpl-logo}

  \[
    \underbrace{4 \times 2}_{\text{\scriptsize 软件学院}}
      + \underbrace{3 \times 6}_{\text{技术科学试验班}}
      + \underbrace{1}_{\text{\blue{\bf 苏州校区重修班}}} = 27 \text{ 名助教}
  \]
\end{frame}
%%%%%%%%%%%%%%%%%%%%

%%%%%%%%%%%%%%%%%%%%
\begin{frame}{}
  \begin{center}
    顺利完成三次机试:\green{\bf 10 月 29 日}、\green{\bf 12 月 09 日}、\blue{\bf 1 月 03 日}

    \vspace{0.20cm}
    \fig{width = 0.95\textwidth}{figs/cpl-scores}
    \vspace{0.30cm}

    感谢各位老师的支持与帮助
  \end{center}
\end{frame}
%%%%%%%%%%%%%%%%%%%%

%%%%%%%%%%%%%%%%%%%%
% \begin{frame}{}
%   \begin{center}
%     与去年相同, 每周安排 9 次答疑
%     \fig{width = 0.70\textwidth}{figs/cpl-qa}
%   \end{center}
% \end{frame}
%%%%%%%%%%%%%%%%%%%%

%%%%%%%%%%%%%%%%%%%%
\begin{frame}{}
  \begin{center}
    \blue{\bf 新措施 (一):} \red{线上}答疑收集表 (积极踊跃; 拯救``社恐患者'')
    \fig{width = 1.00\textwidth}{figs/qa-table}
  \end{center}
\end{frame}
%%%%%%%%%%%%%%%%%%%%

%%%%%%%%%%%%%%%%%%%%
% \begin{frame}{}
%   \fig{width = 1.00\textwidth}{figs/cpl-bilibili-lectures}
% \end{frame}
%%%%%%%%%%%%%%%%%%%%

%%%%%%%%%%%%%%%%%%%%
\begin{frame}{}
  \begin{center}
    \blue{\bf 新措施 (二):} ``短视频''形式专题, 补充知识点
    \fig{width = 0.88\textwidth}{figs/cpl-bilibili-outofclass}
  \end{center}
\end{frame}
%%%%%%%%%%%%%%%%%%%%

%%%%%%%%%%%%%%%%%%%%
\begin{frame}{}
  \begin{center}
    2023 春季, 《编译原理》由\blue{\bf 选修课}改为\red{\bf 专业必修课}

    \vspace{0.30cm}
    \fig{width = 0.40\textwidth}{figs/compiler-logo}
  \end{center}
\end{frame}
%%%%%%%%%%%%%%%%%%%%

%%%%%%%%%%%%%%%%%%%%
\begin{frame}{}
  \begin{center}
    本学期: {作业 (0 分) + \red{\bf 实验 (75 分)} + 期末测试 (25 分)}

    \vspace{0.80cm}

    \textcolor{gray}{上学期: {作业 (15 分) + \red{\bf 实验 (45 分)} + 期末测试 (40 分)}}

    \vspace{1.50cm}
    实验分数高, \blue{\bf 高分段}人数较多

    \vspace{0.60cm}
    下学期考虑调整
  \end{center}
\end{frame}
%%%%%%%%%%%%%%%%%%%%

%%%%%%%%%%%%%%%%%%%%
\begin{frame}{}
  \begin{center}
    \blue{\bf 尽量跟进现代编译器开发原理与实践}:
    \green{加强 LLVM}, \red{引入 RISC-V}

    \fig{width = 0.50\textwidth}{figs/llvm-riscv}
    没有合适的教材, 下学期计划编写《编译原理》课程\blue{\bf 讲义}
  \end{center}
\end{frame}
%%%%%%%%%%%%%%%%%%%%

%%%%%%%%%%%%%%%%%%%%
\begin{frame}{}
  \begin{center}
    逐步对\red{\bf 校外}开放《C 语言程序设计基础》与《编译原理》课程资源

    \vspace{0.30cm}
    \fig{width = 1.00\textwidth}{figs/oj-compilers}

    \vspace{-0.50cm}
    \fig{width = 0.50\textwidth}{figs/impact}
    \vspace{-0.50cm}
  \end{center}
\end{frame}
%%%%%%%%%%%%%%%%%%%%